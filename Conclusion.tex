This thesis aimed to address the main problems affecting today's standard certification processes when applied to highly dynamic and pervasive systems. More specifically, the objective was to take a first step toward the development of a certification scheme that would drastically reduce the redundancy and the time and resources needed by a system to obtain a certificate after some configuration change. The real-world-inspired scenario in Chapter 4 shows what the process would look like and how the proposed scheme would drastically decrease the effort needed compared to a traditional certification process such as the Common Criteria.

We introduced the scoring system and the trigger component and seamlessly added them to the already existing set of components, showing how they could be remodelled to fit together perfectly. Finally, in our experimentation, the scheme successfully delivered a certificate that could complement the previously obtained ones, avoiding redundancies in the process. However, the approach taken in this thesis might raise questions regarding the needed data and taxonomies during the manual review phase that, even if reduced, still represents a core element of the procedure. We note that this problem is tightly related to how standard approaches (e.g. Common Criteria) dealt with similar features developing mechanisms like the Protection Profiles, which are little more than a taxonomy regularly updated with new systems' data.

Further research is needed to develop the aspects that remained out of the focus of this thesis, such as the trigger component implementation, the automatisms connecting each phase with the next one and the data organization for the manual review phase. Moreover, the proposed scheme is intended to be fully integrated with the existing certification schemes since its main purpose is not to substitute them but to complement them, and additional research should be done to allow for it.

The work we presented first addressed how the issues relative to highly dynamic systems could be solved, allowing for simpler re-certification practices. The level of assurance granted by this kind of approach could finally allow systems manufacturers and administrators to improve their services and not worry about releasing security patches or upgrading their systems, causing the complete invalidation of certificates obtained through complex procedures. 