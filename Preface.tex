In recent years, the continuous research and development of increasingly dynamic and pervasive architectures led to new, more distributed and efficient services. However, this development has been met with a decreasing general (perceived) trustworthiness due to the direction needed to provide a better quality of service. The evolution from monolithic and static web services to the micro-services provided by the cloud allowed the understanding of how dynamic infrastructures drastically increase the general flexibility of a system; more specifically, service providers need not worry about the infrastructure but only about the software and end-users can benefit from a better and faster service granted by powerful server machines. Following this line of development, the Edge computing paradigm improved the concept of decentralization, allowing another technology to develop in functionalities, the Internet of Things (IoT). The development path led to networks and systems composed of high numbers of low-power devices, which make it difficult to test and certify their security features. IoT systems heavily rely on the possibility of adding, removing and relocating the numerous devices in their network, which can reach the thousands, without ever stopping the entire system; such configuration changes often imply relative software updates. On the other hand, ICT systems prove their assurance by means of certification, which implies long and heavy processes to release a certificate, which can easily be invalidated by any change in the configuration of the certified product.

The goal of this thesis is to provide a first approach to this problem, allowing highly dynamic systems to perform small changes in their configuration without invalidating the whole certificate, and quickly certifying only the needed properties with the minimum necessary effort.